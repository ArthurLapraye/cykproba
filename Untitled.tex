\documentclass[a4paper,11pt]{article}
\usepackage[utf8]{inputenc}
\usepackage[french]{babel}


\begin{document}

Le corpus étant syntaxiquement structuré, on a pu à l'aide d'un outil, PLY, écrire une grammaire afin d'extraire les informations dont on avait besoin. 

PLY, l'équivalent python de LexYacc en C, permet de faire du parsing LALR. Il suffit donc, de lui donner un lexique - ensemble d'expressions régulières, 
ainsi qu'une syntaxe - ensemble de productions.

La syntaxe est différente d'un parsing classique car à chaque production, une ou plusieurs instructions seront données. On parle alors de sémantique dirigée par la syntaxe. 
Pour notre sujet, l'information sémantique sera principalement l'instantiation de différents objets que nous avons définis dans notre typologie (voir annexe 1).

Dans une logique objet, on a tenté de monter une typologie des productions. Lorsqu'un production est instantiée, deux dictionnaires se remplissent, un propre aux parties gauches 
de règles, tandis que l'autre aux productions elles-mêmes. Ces deux dictionnaires sont les compteurs d'occurences qui nous permettront de calculer les probabilités.

Mathématiquement, une grammaire hors-contexte probabiliste (PCFG) se définit en un quintuplet, soit: $$G = < N, T, A, P, \rho >$$Notre implémentation modifie quelque peu la représentation mathématique. 
$\rho$ n'est pas propre à la grammaire mais aux productions et ce sont ces productions contenant $\rho$ qui seront propres à une PCFG.

Terminal et Nonterminal, sont des types dans notre implémentation, de manière identique aux productions, à chaque instantiation de ces deux types, un ensemble (set) va se remplir, cette fois ci sans compteur.

Lorsque l'extraction se termine, il ne nous reste plus qu'à récupérer les trois ensembles - Terminal, Nonterminal, Productions - pour les intégrer dans un objet Grammaire.

L'axiome de notre grammaire est fixé par notre grammaire, toutefois, on verra par la suite qu'il existe plusieurs symboles "axiome" de la grammaire.

La grammaire constituée sera ensuite sauvegardée dans un fichier au format pickle, à la fois pour un gain de temps à l'accès des données mais aussi pour la place du fichier en mémoire.

\end{document}
