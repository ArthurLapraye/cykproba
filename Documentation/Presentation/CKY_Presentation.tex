\documentclass{beamer}
\usetheme{Boadilla}
\usepackage[utf8]{inputenc} 
\usepackage{amsmath}
\usepackage[T1]{fontenc}
%\usepackage{lmodern}
\usepackage{graphicx}
\usepackage{algpseudocode}
\usepackage{algorithm}



\begin{document}

\title{CKY Probabilisé}  %PAGE 1
\author{Viegas\\Lapraye\\Lévêque}

\institute{Paris Diderot VII}
\date{\today}

\addtobeamertemplate{footline}{\insertframenumber/\inserttotalframenumber}

\begin{frame}
 \maketitle
\end{frame}



\begin{frame} % PAGE 2
\frametitle{Plan}
\end{frame}

\begin{frame}
\frametitle{Extraction de la grammaire}

\begin{itemize}
 \item<1-3> Le corpus
 \item<2-3> 
\end{itemize}

 
\end{frame}

\begin{frame}
 \frametitle{CYK}
 Binarisation 
 intégrer les probas dans CYK
 minimiser la grammaire
 les traits morphosyntaxiques ( FCFG, NT en plus ? )
 gestion des mots inconnus (lemmatisation, lissage...)
 
\end{frame}

\begin{frame}
\frametitle{Evaluation}
 Reconversion plus comparaison au analyse gold
? neutralisation de la grammaire
comment retient on le bon parsing ?
\end{frame}


\frametitle{Forme Normale de Chomsky}
\begin{algorithm}
\caption{CKY}
\label{cky}
	\small{\begin{algorithmic}
		\Function{CKY}{$w[1..n]$, $G: < Q, X, P, \rho>$} \Comment w est un mot à parser
			\State $M \gets tab[1..m];$ \Comment m est la longueur du mot
			\For{$i \gets 1, |m|$} \Comment boucle qui remplit la diagonale
				\State $M[i, i] \gets \{ N / N \rightarrow m[i]\};$
				\For{$d \gets 1, |m|-1$}
					\For{$i \gets 1, |m|-d$}
						\State $j \gets i+d$
						\State $M[i, j] \gets \{\}$
						\ForAll{$k \gets i, j-1$}
							\ForAll{$B \in M[i, k] and C \in M[k+1, j]$}
								\ForAll{$N \in \{ N / N \rightarrow BC \in P \}$}
									\State $M[i, j] \gets M[i, j] \cup N;$
								\EndFor
							\EndFor
						\EndFor
					\EndFor
				\EndFor
			\EndFor
			\State \Return $M$
		\EndFunction
	\end{algorithmic}}
\end{algorithm}
 

\begin{algorithm}
\caption{CNF}
\label{cnf}
	\small{\begin{algorithmic}
		\Function{ChomskyNormalForm}{$G: < Q, X, S, P>$} \Comment Q: terminaux, X: non-terminaux, S: axiome, P: productions
			\State $P' \gets \{\};$
			\State $X' \gets X;$
			\If{$S \in r.h.s\ de\ p \in P$}
				\State $S' \gets S;$
				\State $P' \gets P' \cup S' \rightarrow S;$
			\EndIf
			\State $P' \gets \Call{RemoveNullProductions}{$P'$}$
			\State $P' \gets \Call{RemoveUnitProductions}{$P'$}$
			\For{$p \in \{ A \rightarrow B_1..B_n/ n > 2\}$}
				\State $C \gets B_2..B_n;$
				\State $P' \gets P' \cup A \rightarrow B_1C;$
			\EndFor
			\For{$p \in \{ A \rightarrow aB, a \in Q \}$}
				\State $A_a \gets a;$
				\State $P' \gets P' \cup \{ A \rightarrow A_aB, A_a \rightarrow a \}$
			\EndFor
			\State \Return $G': < Q, X', S', P' >$
		\EndFunction
	\end{algorithmic}}
\end{algorithm}


%\begin{frame}
%\frametitle{Modèle de langue PCFG}
%\framesubtitle{}
%
%Contenu du transparent.
%
%\end{frame}
%
%
%\begin{frame}
%\frametitle{}
%\framesubtitle{}
%
%Contenu du transparent.
%
%\end{frame}


\end{document}