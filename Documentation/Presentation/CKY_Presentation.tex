\documentclass{beamer}
\usepackage[utf8]{inputenc} 
\usepackage{amsmath}
\usepackage[T1]{fontenc}
%\usepackage{lmodern}
\usepackage{graphicx}
\usepackage{algpseudocode}
\usepackage{algorithm}

\begin{document}

\title{CKY Probabilisé}
\maketitle

\begin{frame}
\frametitle{Forme Normale de Chomsky}
\framesubtitle{}

Contenu du transparent.

\end{frame}



\begin{algorithm}
\caption{CKY}
\label{cky}
	\begin{algorithmic}
		\Function{CKY}{$w[1..n]$, $G: < Q, X, P, \rho>$}
			\State $T[1..n,1..n];$
			\State $S \gets 1;$
			\For{$t \gets 1, n$}
				\State $x \gets t-1;$
				\For{$j \gets 1, |Q|$}
					\State $efibzouv$
				\EndFor
			\EndFor
		\EndFunction
	\end{algorithmic}
\end{algorithm}
1 - Extraction de la grammaire
2 - Binarisation (CNF, CYK)
	i -    intégrer les probas dans CYK
	ii -   minimiser la grammaire
	iii -  les traits morphosyntaxiques ( FCFG, NT en plus ? )
	iv -  gestion des mots inconnus (lemmatisation, lissage...)
	v - 
3 - Evaluation
	Reconversion plus comparaison au analyse gold
	? neutralisation de la grammaire
	comment retient on le bon parsing ?
			soit n-premiers
			soit seuil

\begin{frame}
\frametitle{Modèle de langue PCFG}
\framesubtitle{}

Contenu du transparent.

\end{frame}


\begin{frame}
\frametitle{}
\framesubtitle{}

Contenu du transparent.

\end{frame}


\end{document}