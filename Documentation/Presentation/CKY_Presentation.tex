\documentclass{beamer}
\usetheme{Boadilla}
\usepackage[utf8]{inputenc} 
\usepackage{amsmath}
\usepackage[T1]{fontenc}
%\usepackage{lmodern}
\usepackage{graphicx}
\usepackage{algpseudocode}
\usepackage{algorithm}



\begin{document}

\title{CKY Probabiliste}  %PAGE 1
\author{Viegas\\Lapraye\\Lévêque}

\institute{Paris Diderot VII}
\date{\today}

\addtobeamertemplate{footline}{\insertframenumber/\inserttotalframenumber}

\begin{frame}
 \maketitle
\end{frame}



\begin{frame} % PAGE 2
\frametitle{Plan}
\end{frame}

\begin{frame}
\frametitle{Extraction de la grammaire}

\begin{itemize}
 \item<1-3> Le corpus
 \item<2-3> 
\end{itemize}

 
\end{frame}

\begin{frame}
 \frametitle{CYK}
 Binarisation 
 intégrer les probas dans CYK
 minimiser la grammaire
 les traits morphosyntaxiques ( FCFG, NT en plus ? )
 gestion des mots inconnus (lemmatisation, lissage...)
 
\end{frame}

\begin{frame}
\frametitle{Evaluation}
 Reconversion plus comparaison au analyse gold
? neutralisation de la grammaire
comment retient on le bon parsing ?
\end{frame}





%\begin{frame}
%\frametitle{Modèle de langue PCFG}
%\framesubtitle{}
%
%Contenu du transparent.
%
%\end{frame}
%
%
%\begin{frame}
%\frametitle{}
%\framesubtitle{}
%
%Contenu du transparent.
%
%\end{frame}


\end{document}
