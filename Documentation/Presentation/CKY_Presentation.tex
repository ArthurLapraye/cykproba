\documentclass{beamer}
\usetheme{Boadilla}
\usepackage[utf8]{inputenc} 
\usepackage{amsmath}
\usepackage[T1]{fontenc}
%\usepackage{lmodern}
\usepackage{graphicx}
\usepackage{algpseudocode}
\usepackage{algorithm}
\usepackage{biblatex}


\begin{document}

\title{CYK Probabiliste}  %PAGE 1
\author{Viegas\\Lapraye\\Lévêque}

\institute{Paris Diderot VII}
\date{\today}

\addtobeamertemplate{footline}{\insertframenumber/\inserttotalframenumber}

\begin{frame}
 \maketitle
\end{frame}



% \begin{frame} % PAGE 2
% \frametitle{Plan}
% \tableofcontents
% \end{frame}

\begin{frame}
\frametitle{Extraction de la grammaire}

\begin{itemize}
 \item<1-3> Le corpus Sequoia
 \item<2-3> Extraction de probabilités
 \item<3-3> Lemmatisation
\end{itemize}

 
\end{frame}

\begin{frame}
 \frametitle{CYK}
 \begin{itemize}
  \item<1-3>{Passage en forme normale de Chomsky}
  \item<2-3>{Intégrer les probas dans CYK}
  \begin{itemize}
   \item<2-3>{Gestion des analyses multiples}
  \end{itemize}
  \item<3>{Tokens inconnus}
 
 \end{itemize}

 
\end{frame}

\begin{frame}
\frametitle{Evaluation}
 \begin{itemize}
  
  \item<1-3>{Tests manuels}
  
  \item<2-3>{Comparaison avec l'analyse gold}
  \begin{itemize}
   \item<3-3>{Nécessite une conversion de l'arbre vers la grammaire originale}
  \end{itemize}

 \end{itemize}

\end{frame}



\begin{frame}[allowframebreaks]{References}
  \begin{thebibliography}{10}    
  \beamertemplatebookbibitems
  \bibitem{roark}
    Brian Roark, Richard Sproat.
    \newblock {\em Computational Approaches to Morphology and Syntax}.
    \newblock Oxford University Press, 2007.
  \beamertemplatearticlebibitems
  \bibitem{parserEval}
    Mariana Romanyshyn, Vsevolod Dyomkin.
    \newblock {\em The Dirty Little Secret of Constituency Parser Evaluation}, 2014.
    \newblock {http://tech.grammarly.com/blog/posts/The-Dirty-Little-Secret-of-Constituency-Parser-Evaluation.html}
    
  \end{thebibliography}
\end{frame}


\end{document}
