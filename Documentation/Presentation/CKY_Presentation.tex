\documentclass{beamer}
\usetheme{Boadilla}
\usepackage[utf8]{inputenc} 
\usepackage{amsmath}
\usepackage[T1]{fontenc}
%\usepackage{lmodern}
\usepackage{graphicx}
\usepackage{algpseudocode}
\usepackage{algorithm}



\begin{document}

\title{CKY Probabiliste}  %PAGE 1
\author{Viegas\\Lapraye\\Lévêque}

\institute{Paris Diderot VII}
\date{\today}

\addtobeamertemplate{footline}{\insertframenumber/\inserttotalframenumber}

\begin{frame}
 \maketitle
\end{frame}



% \begin{frame} % PAGE 2
% \frametitle{Plan}
% \tableofcontents
% \end{frame}

\begin{frame}
\frametitle{Extraction de la grammaire}

\begin{itemize}
 \item<1-3> Le corpus Sequoia
 \item<2-3> Extraction de probabilités
 \item<3-3> Lemmatisation
\end{itemize}

 
\end{frame}

\begin{frame}
 \frametitle{CYK}
 \begin{itemize}
  \item<1-3>{Passage en forme normale de Chomsky}
  \item<2-3>{Intégrer les probas dans CYK}
  \begin{itemize}
   \item<2-3>{Gestion des analyses multiples}
  \end{itemize}
  \item<3>{Tokens inconnus}
 
 \end{itemize}

 
\end{frame}

\begin{frame}
\frametitle{Evaluation}
 \begin{itemize}
  
  \item<1-3>{Tests manuels}
  
  \item{Comparaison avec l'analyse gold}
  \begin{itemize}
   \item{Nécessite une conversion de l'arbre vers la grammaire originale}
  \end{itemize}

 \end{itemize}

\end{frame}



\begin{frame}[allowframebreaks]
        \frametitle{References}
        %\bibliographystyle{amsalpha}
       % \bibliography{../bib_files/jabrefmaster.bib}
\end{frame}

%\begin{frame}
%\frametitle{Modèle de langue PCFG}
%\framesubtitle{}
%
%Contenu du transparent.
%
%\end{frame}
%
%
%\begin{frame}
%\frametitle{}
%\framesubtitle{}
%
%Contenu du transparent.
%
%\end{frame}


\end{document}
